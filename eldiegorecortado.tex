% LaTeXar con :
%  pdflatex eldiego.tex
%"The PDF file may contain up to 25 pages of reference material, single-sided, letter or A4 size, with text and illustrations readable by a person with correctable eyesight without magnification from a distance of 1/2 meter."
\input{preamble.margenes.chicos.tex}
\begin{document}
\def\title{Caloventor en Dos - Universidad Nacional de Rosario}
\section{algorithm}%%%%%%%%%%%%%%%%%%ALGORITHM%%%%%%%%%%%%%%%%%%
\input{algorithm.tex}

\section{Estructuras}%%%%%%%%%%%%%%%%%%ESTRUCTURAS%%%%%%%%%%%%%%%%%%
\subsection{RMQ (static)}
Dado un arreglo y una operacion asociativa \emph{idempotente}, get(i, j) opera sobre el rango [i, j). Restriccion: LVL $\ge$ ceil(logn); Usar [ ] para llenar arreglo y luego build().
\cppfile{estructuras/rmq.static.cpp}
\subsection{RMQ (dynamic)}
\cppfile{estructuras/rmq.dynamic.cpp}
\subsection{RMQ (lazy)}
\cppfile{estructuras/rmq.lazy.cpp}
\subsection{RMQ (persistente)}
\cppfile[23-53]{estructuras/rmq.persistent.cpp}
\subsection{Fenwick Tree}
\cppfile{estructuras/fenwick.cpp}
\subsection{Union Find}
\cppfile{grafos/union.find.cpp}
\subsection{Disjoint Intervals}
\cppfile{estructuras/disjoint.intervals.cpp}
\subsection{RMQ (2D)}
\cppfile{estructuras/rmq.2d.cpp}
\subsection{Big Int}
\cppfile{estructuras/bigint.villa.cpp}
\subsection{HashTables}
\cppfile[16-30]{hash.cpp}
\subsection{Modnum}
\cppfile{estructuras/mnum.cpp}
\subsection{Treap para set}
\cppfile[16-89]{estructuras/treap.cpp}
\subsection{Treap para arreglo}
\cppfile[17-87]{estructuras/treaparr.cpp}
\subsection{Convex Hull Trick}
\cppfile[18-55]{estructuras/convexhull.trick.cpp}
\subsection{Convex Hull Trick (Dynamic)}
\cppfile[17-56]{estructuras/convexhull.trick.dyn.cpp}
%\subsection{Gain-Cost Set}
%\cppfile[20-43]{estructuras/gain-cost.set.cpp}
\subsection{Set con busq binaria}
\cppfile[17-26]{estructuras/order.tree.cpp}


\section{Algos}%%%%%%%%%%%%%%%%%%ALGORITMOS%%%%%%%%%%%%%%%%%%%%%%%%%%
\subsection{Longest Increasing Subsecuence}
\cppfile[14-42]{algos/lis.cpp}
\subsection{Alpha-Beta prunning}
\cppfile{algos/alphabeta.cpp}
\subsection{Mo's algorithm}
\cppfile{algos/mosalgorithm.cpp}


\section{Strings}%%%%%%%%%%%%%%%%%%STRINGS%%%%%%%%%%%%%%%%%%%%%%%%%%
\subsection{Manacher}
\cppfile[18-37]{string/manacher.cpp}
\subsection{KMP}
\cppfile[21-38]{string/kmp.cpp}
\subsection{Trie}
\cppfile{string/trie.cpp}
%\subsection{Suffix Array (corto, nlog2n)}
%\cppfile[12-26]{string/suffix.array.short.cpp}
\subsection{Suffix Array (largo, nlogn)}
\cppfile[-34]{string/suffix.array.cpp}
\subsection{String Matching With Suffix Array}
\cppfile[37-58]{string/suffix.array.cpp}
\subsection{LCP (Longest Common Prefix)}
\cppfile[60-75]{string/suffix.array.cpp}
\subsection{Corasick}
\cppfile[16-55]{string/corasick.cpp}
\subsection{Suffix Automaton}
\cppfile[16-61]{string/suffix.automaton.cpp}
\subsection{Z Function}
\cppfile[17-27]{string/zfunction.cpp}


\section{Geometria}%%%%%%%%%%%%%%%%%%GEOMETRIA%%%%%%%%%%%%%%%%%%%%%%
\subsection{Punto}
\cppfile[2-33]{geometria/pto.cpp}
\subsection{Orden radial de puntos}
\cppfile{geometria/orden.radial.cpp}
\subsection{Line}
\cppfile{geometria/line.cpp}
\subsection{Segment}
\cppfile{geometria/segm.cpp}
%\subsection{Rectangle}
%n\cppfile{geometria/rect.cpp}
\subsection{Polygon Area}
\cppfile{geometria/area.cpp}
\subsection{Circle}
\cppfile{geometria/circle.cpp}
\subsection{Point in Poly}
\cppfile{geometria/point.in.poly.cpp}
\subsection{Point in Convex Poly log(n)}
\cppfile{geometria/point.in.convex.poly.cpp}
%\subsection{Convex Check CHECK}
%\cppfile{geometria/convex.check.cpp}
\subsection{Convex Hull}
\cppfile{geometria/convex.hull.cpp}
\subsection{Cut Polygon}
\cppfile{geometria/cut.polygon.cpp}
\subsection{Bresenham}
\cppfile{geometria/bresenham.cpp}
%\subsection{Rotate Matrix}
%\cppfile{geometria/rotate.cpp}
\subsection{Interseccion de Circulos en n3log(n)}
\cppfile{geometria/int.circs.cpp}


\section{Math}%%%%%%%%%%%%%%%%%%MATH%%%%%%%%%%%%%%%%%%%%%%%%%%%%%%%%
\subsection{Identidades}
{
$\sum_{i=0}^n\binom{n}{i}=2^n$

$\sum_{i=0}^n i\binom{n}{i}=n*2^{n-1}$

$\sum_{i=m}^n i = \frac{n(n+1)}{2} - \frac{m(m-1)}{2} = \frac{(n+1-m)(n+m)}{2}$

$\sum_{i=0}^n i = \sum_{i=1}^n i = \frac{n(n+1)}{2}$

$\sum_{i=0}^n i^2 = \frac{n(n+1)(2n+1)}{6} = \frac{n^3}{3} + \frac{n^2}{2} + \frac{n}{6}$

$\sum_{i=0}^n i(i-1) = \frac{8}{6}(\frac{n}{2})(\frac{n}{2}+1)(n+1)$ (doubles) $\rightarrow$ Sino ver caso impar y par

$\sum_{i=0}^n i^3 = \left(\frac{n(n+1)}{2}\right)^2 = \frac{n^4}{4} + \frac{n^3}{2} + \frac{n^2}{4} = \left[\sum_{i=1}^n i\right]^2$

$\sum_{i=0}^n i^4 = \frac{n(n+1)(2n+1)(3n^2+3n-1)}{30} = \frac{n^5}{5} + \frac{n^4}{2} + \frac{n^3}{3} - \frac{n}{30}$

$\sum_{i=0}^n i^p = \frac{(n+1)^{p+1}}{p+1} + \sum_{k=1}^p\frac{B_k}{p-k+1}{p\choose k}(n+1)^{p-k+1}$

$r=e-v+k+1$

Teorema de Pick: (Area, puntos interiores y puntos en el borde)

$A=I+\frac{B}{2}-1$


}%
\subsection{Ec. Caracteristica}
$a_0T(n)+a_1T(n-1)+...+a_kT(n-k)=0$

$p(x)=a_0 x^k + a_1 x^{k-1} + ... + a_k$

Sean $r_1,r_2,...,r_q$ las raíces distintas, de mult. $m_1, m_2, ..., m_q$

$T(n)=\sum_{i=1}^q{\sum_{j=0}^{m_i - 1}c_{ij} n^j r_i^n}$

Las constantes $c_{ij}$ se determinan por los casos base.
\subsection{Combinatorio}
\cppfile{math/combinatorio.cpp}
\subsection{Exp. de Numeros Mod.}
\cppfile[2]{math/exp.mod.cpp}
\subsection{Exp. de Matrices}
\cppfile{math/exp.mat.cpp}
\subsection{Matrices y determinante $O(n^3)$}
\cppfile[17-52]{math/determinante.cpp}
\subsection{Teorema Chino del Resto}
$$y=\sum_{j=1}^n (x_j*(\prod_{i=1, i\neq j}^n m_i)_{m_j}^{-1}*\prod_{i=1, i\neq j}^n m_i)$$
\subsection{Criba}
\cppfile[19-34
]{math/criba.cpp}
\subsection{Funciones de primos}
Sea $n=\prod{p_i^{k_i}}$, fact(n) genera un map donde a cada $p_i$ le asocia su $k_i$
\cppfile[33-88]{math/func.primos.cpp}
\subsection{Phollard's Rho (rolando)}
\cppfile{math/phollards.rho.villa.cpp}
\subsection{GCD}
\begin{code}
tipo gcd(tipo a, tipo b){return a?gcd(b %a, a):b;}
\end{code}
\subsection{Extended Euclid}
\cppfile{math/extended.euclid.cpp}
\subsection{LCM}
\begin{code}
tipo lcm(tipo a, tipo b){return a / gcd(a,b) * b;}
\end{code}
\subsection{Inversos}
\cppfile[7-16]{math/inversos.cpp}
\subsection{Simpson}
\cppfile{math/simpson.cpp}
\subsection{Fraction}
\cppfile{math/frac.cpp}
\subsection{Polinomio}
\cppfile[16-74]{math/polinomio.cpp}
\subsection{Ec. Lineales}
\cppfile[29-62]{math/eclineales.cpp}
\subsection{FFT}
\cppfile[16-65]{math/fft.cpp}
\subsection{Tablas y cotas (Primos, Divisores, Factoriales, etc)}
%\subsubsection{
 
\paragraph{Cantidad de primos menores que $10^n$}\ \\
$\pi(10^1)$ = 4 ;
$\pi(10^2)$ = 25 ;
$\pi(10^3)$ = 168 ;
$\pi(10^4)$ = 1229 ;
$\pi(10^5)$ = 9592 \\
$\pi(10^6)$ = 78.498 ;
$\pi(10^7)$ = 664.579 ;
$\pi(10^8)$ = 5.761.455 ;
$\pi(10^9)$ = 50.847.534 \\
$\pi(10^{10})$ = 455.052,511 ;
$\pi(10^{11})$ = 4.118.054.813 ;
$\pi(10^{12})$ = 37.607.912.018% ;
%
% Fuente: http://primes.utm.edu/howmany.shtml#table
%
%

%\subsubsection{Divisores}
\paragraph{Divisores} \ \\
Cantidad de divisores ($\sigma_0$) para \emph{algunos} $n / \neg\exists n'<n, \sigma_0(n') \geqslant \sigma_0(n)$ \\
$\sigma_0(60)$ = 12 ; $\sigma_0(120)$ = 16 ; $\sigma_0(180)$ = 18 ; $\sigma_0(240)$ = 20 ; $\sigma_0(360)$ = 24 \\
$\sigma_0(720)$ = 30 ; $\sigma_0(840)$ = 32 ; $\sigma_0(1260)$ = 36 ; $\sigma_0(1680)$ = 40 ; $\sigma_0(10080)$ = 72 \\ $\sigma_0(15120)$ = 80 ; $\sigma_0(50400)$ = 108 ; $\sigma_0(83160)$ = 128 ; $\sigma_0(110880)$ = 144 \\
$\sigma_0(498960)$ = 200 ; $\sigma_0(554400)$ = 216 ; $\sigma_0(1081080)$ = 256 ; $\sigma_0(1441440)$ = 288  $\sigma_0(4324320)$ = 384 ; $\sigma_0(8648640)$ = 448

\section{Grafos}%%%%%%%%%%%%%%%%%%GRAFOS%%%%%%%%%%%%%%%%%%%%%%%%%%%%
\subsection{Dijkstra}
\cppfile[6-23]{grafos/dijkstra.cpp}
\subsection{Bellman-Ford}
\cppfile{grafos/bellman.ford.cpp}
\subsection{Floyd-Warshall}
\cppfile{grafos/floyd.warshall.cpp}
\subsection{Kruskal}
\cppfile[27-41]{grafos/kruskal.cpp}
\subsection{Prim}
\cppfile[23-40]{grafos/prim.cpp}
\subsection{2-SAT + Tarjan SCC}
\cppfile{grafos/2sat.cpp}
\subsection{Articulation Points}
\cppfile{grafos/articulaciones.cpp}
\subsection{Comp. Biconexas y Puentes}
\cppfile[28-76]{grafos/biconexas.bridge.cpp}
\subsection{LCA + Climb}
\cppfile{grafos/lca.climb.cpp}
\subsection{Heavy Light Decomposition}
\cppfile[21-56]{grafos/heavylight.cpp}
\subsection{Centroid Decomposition}
\cppfile[17-36]{grafos/centroid.cpp}
\subsection{Euler Cycle}
\cppfile{grafos/euler.cpp}
\subsection{Diametro árbol}
\cppfile[17-41]{grafos/diametro.cpp}
\subsection{Chu-liu}
\cppfile{grafos/chuliu.villa.cpp}
\subsection{Hungarian}
\cppfile{grafos/hungarian.villa.cpp}
\subsection{Dynamic Conectivity}
\cppfile[17-78]{grafos/dynamic.conectivity.cpp}


\section{Network Flow}%%%%%%%%%%%%%%%%%%FLOW%%%%%%%%%%%%%%%%%%%%%%%%%%%%
\subsection{Dinic}
\cppfile[12-66]{flow/dinic.cpp}
%\subsection{Konig}
%\cppfile[98-122]{flow/konig.cpp}
%\subsection{Edmonds Karp's}
%\cppfile{flow/edmonds.karps.cpp}
%\subsection{Push-Relabel O(N3)}
%\cppfile{flow/push.relabel.cpp}
\subsection{Min-cost Max-flow}
\cppfile[16-66]{flow/min.cost.max.flow.cpp}


\section{Template}%%%%%%%%%%%%%%%%%%TEMPLATE%%%%%%%%%%%%%%%%
\cppfile{template.cpp}


\section{Ayudamemoria}%%%%%%%%%%%%%%%%%%AYUDAMEMORIA%%%%%%%%%%%%%%%%
\subsection*{Leer hasta fin de linea}
\begin{code}
#include <sstream>
//hacer cin.ignore() antes de getline()
while(getline(cin, line)){
   	 istringstream is(line);
   	 while(is >> X)
   		 cout << X << " ";
   	 cout << endl;
}
\end{code}
\subsection*{Expandir pila}
\begin{code}
#include <sys/resource.h>
rlimit rl;
getrlimit(RLIMIT_STACK, &rl);
rl.rlim_cur=1024L*1024L*256L;//256mb
setrlimit(RLIMIT_STACK, &rl);
\end{code}
\subsection*{Iterar subconjunto}
\begin{code}
for(int sbm=bm; sbm; sbm=(sbm-1)&bm)
\end{code}
\end{document}
