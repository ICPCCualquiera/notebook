\documentclass[10pt,landscape,twocolumn,a4paper,notitlepage]{article}
\usepackage{hyperref}
\usepackage[spanish, activeacute]{babel}
\usepackage[utf8]{inputenc}
\usepackage{fancyhdr}
\usepackage{lastpage}
\usepackage{listings}
\usepackage{amssymb}
\usepackage[usenames,dvipsnames]{color}
\usepackage{graphicx}
\usepackage{wrapfig}
\usepackage{amsmath}
\usepackage{makeidx}
\usepackage{color}
\usepackage{currfile}
\usepackage{xcolor}

% Margenes
\setlength{\columnsep}{0.4in}
\setlength{\columnseprule}{0.1pt}
\addtolength{\textheight}{2.9in}
\addtolength{\topmargin}{-1.2in}
\addtolength{\textwidth}{1.8in}
\addtolength{\oddsidemargin}{-0.9in}
\setlength{\headsep}{0.08in}
\setlength{\parskip}{0in}
\setlength{\headheight}{15pt}
\setlength{\parindent}{0mm}

% Encabezado de pagina
\pagestyle{fancy}
\fancyhead[LO]{\textbf{\title}}
\fancyhead[C]{\leftmark}
\fancyhead[RO]{P\'agina \thepage\ de \pageref{LastPage}}

% Mostrar codigo
\lstloadlanguages{C++}

\lstnewenvironment{code}
	{\csname lst@SetFirstLabel\endcsname}
	{\csname lst@SaveFirstLabel\endcsname}

\lstset{
    language=C++,
    basicstyle=\small\ttfamily, % Set the font style for the code
    keywordstyle=\color{blue}, % Set the color for keywords
    stringstyle=\color{green!60!black}, % Set the color for strings
    commentstyle=\color{red}, % Set the color for comments
    breaklines=true, % Allow breaking lines within the code
    postbreak=\mbox{\textcolor{red}{$\hookrightarrow$}\space}, % Symbol at the end of a broken line
    showstringspaces=false % Don't show spaces in strings as underscores
}

\begin{document}
\def\title{Notebook}
\tableofcontents\newpage

\section{Setup}
    \textbf{Template}
    \lstinputlisting{\currfiledir template.cpp}
    
    \textbf{Makefile}
    \lstinputlisting{\currfiledir Makefile}
    
    \textbf{comp.sh: Compilar \$1 y mostrar primeras \$2 lineas de error}
    \lstinputlisting{\currfiledir compilar.sh}
    
    \textbf{run.sh: Correr \$1 con el input \$2}
    \lstinputlisting{\currfiledir correr.sh}

\section{STL}
    \textbf{Búsqueda binaria - Primer igual}
\lstinputlisting{\currfiledir primer_igual.cpp}

\textbf{Búsqueda binaria - Último igual}
\lstinputlisting{\currfiledir ultimo_igual.cpp}

\textbf{Búsqueda binaria - Primer mayor}
\lstinputlisting{\currfiledir primer_mayor.cpp}

\textbf{Búsqueda binaria - Último menor}
\lstinputlisting{\currfiledir ultimo_menor.cpp}


    \textbf{Búsqueda binaria - Primer igual}
\lstinputlisting{\currfiledir primer_igual.cpp}

\textbf{Búsqueda binaria - Último igual}
\lstinputlisting{\currfiledir ultimo_igual.cpp}

\textbf{Búsqueda binaria - Primer mayor}
\lstinputlisting{\currfiledir primer_mayor.cpp}

\textbf{Búsqueda binaria - Último menor}
\lstinputlisting{\currfiledir ultimo_menor.cpp}


    \textbf{Búsqueda binaria - Primer igual}
\lstinputlisting{\currfiledir primer_igual.cpp}

\textbf{Búsqueda binaria - Último igual}
\lstinputlisting{\currfiledir ultimo_igual.cpp}

\textbf{Búsqueda binaria - Primer mayor}
\lstinputlisting{\currfiledir primer_mayor.cpp}

\textbf{Búsqueda binaria - Último menor}
\lstinputlisting{\currfiledir ultimo_menor.cpp}


    \textbf{Búsqueda binaria - Primer igual}
\lstinputlisting{\currfiledir primer_igual.cpp}

\textbf{Búsqueda binaria - Último igual}
\lstinputlisting{\currfiledir ultimo_igual.cpp}

\textbf{Búsqueda binaria - Primer mayor}
\lstinputlisting{\currfiledir primer_mayor.cpp}

\textbf{Búsqueda binaria - Último menor}
\lstinputlisting{\currfiledir ultimo_menor.cpp}


\section{Range queries}
    \textbf{Búsqueda binaria - Primer igual}
\lstinputlisting{\currfiledir primer_igual.cpp}

\textbf{Búsqueda binaria - Último igual}
\lstinputlisting{\currfiledir ultimo_igual.cpp}

\textbf{Búsqueda binaria - Primer mayor}
\lstinputlisting{\currfiledir primer_mayor.cpp}

\textbf{Búsqueda binaria - Último menor}
\lstinputlisting{\currfiledir ultimo_menor.cpp}


    \textbf{Búsqueda binaria - Primer igual}
\lstinputlisting{\currfiledir primer_igual.cpp}

\textbf{Búsqueda binaria - Último igual}
\lstinputlisting{\currfiledir ultimo_igual.cpp}

\textbf{Búsqueda binaria - Primer mayor}
\lstinputlisting{\currfiledir primer_mayor.cpp}

\textbf{Búsqueda binaria - Último menor}
\lstinputlisting{\currfiledir ultimo_menor.cpp}


\section{Grafos}
%    \textbf{Toposort con Tarjan}
%    \lstinputlisting{\currfiledir toposort_tarjan.cpp}

    \textbf{Toposort}
    \lstinputlisting{\currfiledir toposort_kahn.cpp}

    \textbf{Bipartite check}
    \lstinputlisting{\currfiledir bipartite_check.cpp}

\section{Programacion Dinamica}
    \textbf{Búsqueda binaria - Primer igual}
\lstinputlisting{\currfiledir primer_igual.cpp}

\textbf{Búsqueda binaria - Último igual}
\lstinputlisting{\currfiledir ultimo_igual.cpp}

\textbf{Búsqueda binaria - Primer mayor}
\lstinputlisting{\currfiledir primer_mayor.cpp}

\textbf{Búsqueda binaria - Último menor}
\lstinputlisting{\currfiledir ultimo_menor.cpp}


\section{Matemática}
    \textbf{Búsqueda binaria - Primer igual}
\lstinputlisting{\currfiledir primer_igual.cpp}

\textbf{Búsqueda binaria - Último igual}
\lstinputlisting{\currfiledir ultimo_igual.cpp}

\textbf{Búsqueda binaria - Primer mayor}
\lstinputlisting{\currfiledir primer_mayor.cpp}

\textbf{Búsqueda binaria - Último menor}
\lstinputlisting{\currfiledir ultimo_menor.cpp}


    \textbf{Búsqueda binaria - Primer igual}
\lstinputlisting{\currfiledir primer_igual.cpp}

\textbf{Búsqueda binaria - Último igual}
\lstinputlisting{\currfiledir ultimo_igual.cpp}

\textbf{Búsqueda binaria - Primer mayor}
\lstinputlisting{\currfiledir primer_mayor.cpp}

\textbf{Búsqueda binaria - Último menor}
\lstinputlisting{\currfiledir ultimo_menor.cpp}


    \textbf{Búsqueda binaria - Primer igual}
\lstinputlisting{\currfiledir primer_igual.cpp}

\textbf{Búsqueda binaria - Último igual}
\lstinputlisting{\currfiledir ultimo_igual.cpp}

\textbf{Búsqueda binaria - Primer mayor}
\lstinputlisting{\currfiledir primer_mayor.cpp}

\textbf{Búsqueda binaria - Último menor}
\lstinputlisting{\currfiledir ultimo_menor.cpp}


\section{Geometria}
    \textbf{Búsqueda binaria - Primer igual}
\lstinputlisting{\currfiledir primer_igual.cpp}

\textbf{Búsqueda binaria - Último igual}
\lstinputlisting{\currfiledir ultimo_igual.cpp}

\textbf{Búsqueda binaria - Primer mayor}
\lstinputlisting{\currfiledir primer_mayor.cpp}

\textbf{Búsqueda binaria - Último menor}
\lstinputlisting{\currfiledir ultimo_menor.cpp}

    \textbf{Búsqueda binaria - Primer igual}
\lstinputlisting{\currfiledir primer_igual.cpp}

\textbf{Búsqueda binaria - Último igual}
\lstinputlisting{\currfiledir ultimo_igual.cpp}

\textbf{Búsqueda binaria - Primer mayor}
\lstinputlisting{\currfiledir primer_mayor.cpp}

\textbf{Búsqueda binaria - Último menor}
\lstinputlisting{\currfiledir ultimo_menor.cpp}

    \textbf{Búsqueda binaria - Primer igual}
\lstinputlisting{\currfiledir primer_igual.cpp}

\textbf{Búsqueda binaria - Último igual}
\lstinputlisting{\currfiledir ultimo_igual.cpp}

\textbf{Búsqueda binaria - Primer mayor}
\lstinputlisting{\currfiledir primer_mayor.cpp}

\textbf{Búsqueda binaria - Último menor}
\lstinputlisting{\currfiledir ultimo_menor.cpp}

    \textbf{Búsqueda binaria - Primer igual}
\lstinputlisting{\currfiledir primer_igual.cpp}

\textbf{Búsqueda binaria - Último igual}
\lstinputlisting{\currfiledir ultimo_igual.cpp}

\textbf{Búsqueda binaria - Primer mayor}
\lstinputlisting{\currfiledir primer_mayor.cpp}

\textbf{Búsqueda binaria - Último menor}
\lstinputlisting{\currfiledir ultimo_menor.cpp}


\section{Estructuras locas}
    \subsection{Disjoint set union}
    \lstinputlisting{\currfiledir dsu.cpp}

    \subsection{Binary trie}
    \lstinputlisting{\currfiledir binary_trie.cpp}

\section{Sin categorizar}
    \textbf{Búsqueda binaria sobre un predicado}
    \lstinputlisting{\currfiledir busqueda_binaria.cpp}

    \textbf{Enumerar subconjuntos de un conjuto con bitmask}
    \lstinputlisting{\currfiledir enumerar_subconjuntos.cpp}

    \textbf{Hashing Rabin Karp}
    \lstinputlisting{\currfiledir rabin_karp.cpp}

\section{Brainstorming}
    \begin{itemize}
        \item Graficar como puntos/grafos
        \item Usar geometria
        \item ¿Que propiedades debe cumplir una solución?
        \item ¿Existen varias soluciones? ¿Hay una forma canónica?
        \item ¿Hay elecciones independientes?
        \item Pensarlo al revez
        \item ¿El proceso es parecido a un algoritmo conocido?
        \item Si se busca calcular $f(x)$ para todo $x$, calcular cuánto contribuye $x$ a $f(y)$ para los otros $y$
        \item Definiciones e identidades: ¿\textit{que significa} que un array sea palindromo? (ejemplo)
    \end{itemize}

\end{document}