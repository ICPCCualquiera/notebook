\documentclass[10pt,landscape,twocolumn,a4paper,notitlepage]{article}
\usepackage{hyperref}
\usepackage[spanish, activeacute]{babel}
\usepackage[utf8]{inputenc}
\usepackage{fancyhdr}
\usepackage{lastpage}
\usepackage{listings}
\usepackage{amssymb}
\usepackage[usenames,dvipsnames]{color}
\usepackage{graphicx}
\usepackage{wrapfig}
\usepackage{amsmath}
\usepackage{makeidx}
\usepackage{color}
\usepackage{currfile}
\usepackage{xcolor}

% Margenes
\setlength{\columnsep}{0.4in}
\setlength{\columnseprule}{0.1pt}
\addtolength{\textheight}{2.9in}
\addtolength{\topmargin}{-1.2in}
\addtolength{\textwidth}{1.8in}
\addtolength{\oddsidemargin}{-0.9in}
\setlength{\headsep}{0.08in}
\setlength{\parskip}{0in}
\setlength{\headheight}{15pt}
\setlength{\parindent}{0mm}

% Encabezado de pagina
\pagestyle{fancy}
\fancyhead[LO]{\textbf{\title}}
\fancyhead[C]{\leftmark}
\fancyhead[RO]{P\'agina \thepage\ de \pageref{LastPage}}

% Mostrar codigo
\lstloadlanguages{C++}

\lstnewenvironment{code}
	{\csname lst@SetFirstLabel\endcsname}
	{\csname lst@SaveFirstLabel\endcsname}

\lstset{
    language=C++,
    basicstyle=\small\ttfamily, % Set the font style for the code
    keywordstyle=\color{blue}, % Set the color for keywords
    stringstyle=\color{green!60!black}, % Set the color for strings
    commentstyle=\color{red}, % Set the color for comments
    breaklines=true, % Allow breaking lines within the code
    postbreak=\mbox{\textcolor{red}{$\hookrightarrow$}\space}, % Symbol at the end of a broken line
    showstringspaces=false % Don't show spaces in strings as underscores
}

\begin{document}
\def\title{Notebook}
\tableofcontents\newpage

\section{Setup}
    \textbf{Template corto}
    \lstinputlisting{\currfiledir template_corto.cpp}

    \textbf{run.sh: Compilar y ejecutar \$1 con archivo input opcional \$2}
    \lstinputlisting{\currfiledir run.sh}
    
    \textbf{Makefile}
    \lstinputlisting{\currfiledir Makefile}
    
    \textbf{compilar.sh: Compilar \$1 y mostrar primeras \$2 lineas de error}
    \lstinputlisting{\currfiledir compilar.sh}

    \textbf{Template completo}
    \lstinputlisting{\currfiledir template_completo.cpp}

\section{STL}
    \subsubsection{Operaciones de conjuntos y modificación}
    \textbf{Funciones que modifican rangos}
    \input{\currfiledir modificar.tex}

    \textbf{Operaciones de conjuntos con vectors ordenados (lineal)}
    \lstinputlisting{\currfiledir conjuntos.cpp}
    \input{\currfiledir conjuntos.tex}

    \subsubsection{Operaciones de conjuntos y modificación}
    \textbf{Funciones que modifican rangos}
    \input{\currfiledir modificar.tex}

    \textbf{Operaciones de conjuntos con vectors ordenados (lineal)}
    \lstinputlisting{\currfiledir conjuntos.cpp}
    \input{\currfiledir conjuntos.tex}

    \subsubsection{Operaciones de conjuntos y modificación}
    \textbf{Funciones que modifican rangos}
    \input{\currfiledir modificar.tex}

    \textbf{Operaciones de conjuntos con vectors ordenados (lineal)}
    \lstinputlisting{\currfiledir conjuntos.cpp}
    \input{\currfiledir conjuntos.tex}

\section{Range queries}
    \textbf{Prefix/dff arrays}
    \lstinputlisting{\currfiledir prefix_diff_arrays.cpp}

    \subsubsection{Operaciones de conjuntos y modificación}
    \textbf{Funciones que modifican rangos}
    \input{\currfiledir modificar.tex}

    \textbf{Operaciones de conjuntos con vectors ordenados (lineal)}
    \lstinputlisting{\currfiledir conjuntos.cpp}
    \input{\currfiledir conjuntos.tex}

    \textbf{Sparse table}
    \lstinputlisting{\currfiledir sparse_table.cpp}

\section{Grafos}
    \textbf{Toposort}
    \lstinputlisting{\currfiledir toposort.cpp}

    \textbf{Bipartite check}
    \lstinputlisting{\currfiledir bipartite_check.cpp}

    \textbf{Encontrar puentes y articulaciones}
    \lstinputlisting{\currfiledir puentes_articulaciones.cpp}

\section{Programacion Dinamica}
    \subsubsection{Operaciones de conjuntos y modificación}
    \textbf{Funciones que modifican rangos}
    \input{\currfiledir modificar.tex}

    \textbf{Operaciones de conjuntos con vectors ordenados (lineal)}
    \lstinputlisting{\currfiledir conjuntos.cpp}
    \input{\currfiledir conjuntos.tex}

    \subsection{Menor ciclo hamiltoniano}
    \lstinputlisting{\currfiledir menor_ciclo_hamiltoniano.cpp}

\section{Matemática}
    \subsubsection{Operaciones de conjuntos y modificación}
    \textbf{Funciones que modifican rangos}
    \input{\currfiledir modificar.tex}

    \textbf{Operaciones de conjuntos con vectors ordenados (lineal)}
    \lstinputlisting{\currfiledir conjuntos.cpp}
    \input{\currfiledir conjuntos.tex}

    \subsection{Sin categorizar}
        \textbf{Test de primalidad}
        \lstinputlisting{\currfiledir primetest.cpp}

        \textbf{Template geometría}
        \lstinputlisting{\currfiledir template_geometria.cpp}

\section{Geometria}
    \subsubsection{Operaciones de conjuntos y modificación}
    \textbf{Funciones que modifican rangos}
    \input{\currfiledir modificar.tex}

    \textbf{Operaciones de conjuntos con vectors ordenados (lineal)}
    \lstinputlisting{\currfiledir conjuntos.cpp}
    \input{\currfiledir conjuntos.tex}
    \subsubsection{Operaciones de conjuntos y modificación}
    \textbf{Funciones que modifican rangos}
    \input{\currfiledir modificar.tex}

    \textbf{Operaciones de conjuntos con vectors ordenados (lineal)}
    \lstinputlisting{\currfiledir conjuntos.cpp}
    \input{\currfiledir conjuntos.tex}
    \subsubsection{Operaciones de conjuntos y modificación}
    \textbf{Funciones que modifican rangos}
    \input{\currfiledir modificar.tex}

    \textbf{Operaciones de conjuntos con vectors ordenados (lineal)}
    \lstinputlisting{\currfiledir conjuntos.cpp}
    \input{\currfiledir conjuntos.tex}
    \subsubsection{Operaciones de conjuntos y modificación}
    \textbf{Funciones que modifican rangos}
    \input{\currfiledir modificar.tex}

    \textbf{Operaciones de conjuntos con vectors ordenados (lineal)}
    \lstinputlisting{\currfiledir conjuntos.cpp}
    \input{\currfiledir conjuntos.tex}

\section{Estructuras locas}
    \subsection{Disjoint set union}
    \lstinputlisting{\currfiledir dsu.cpp}

    \subsection{Binary trie}
    \lstinputlisting{\currfiledir binary_trie.cpp}

\section{Sin categorizar}
    \subsection{Búsqueda binaria sobre un predicado}
    \lstinputlisting{\currfiledir busqueda_binaria.cpp}

    \subsection{Enumerar subconjuntos de un conjuto con bitmask}
    \lstinputlisting{\currfiledir enumerar_subconjuntos.cpp}

    \subsection{Hashing Rabin Karp}
    \lstinputlisting{\currfiledir rabin_karp.cpp}

    \subsection{Lowest common ancestor (LCA)}
    \lstinputlisting{\currfiledir lca.cpp}

    \subsection{Euler tour}
    \lstinputlisting{\currfiledir euler-tour.cpp}

\section{Brainstorming}
    \begin{itemize}
        \item Graficar como puntos/grafos
        \item Usar geometria
        \item ¿Que propiedades debe cumplir una solución?
        \item ¿Existen varias soluciones? ¿Hay una forma canónica?
        \item ¿Hay elecciones independientes?
        \item Pensarlo al revez
        \item ¿El proceso es parecido a un algoritmo conocido?
        \item Si se busca calcular $f(x)$ para todo $x$, calcular cuánto contribuye $x$ a $f(y)$ para los otros $y$
        \item Definiciones e identidades: ¿\textit{que significa} que un array sea palindromo? (ejemplo)
    \end{itemize}

\end{document}