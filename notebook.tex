\documentclass[10pt,landscape,twocolumn,a4paper,notitlepage]{article}
\usepackage{hyperref}
\usepackage[spanish, activeacute]{babel}
\usepackage[utf8]{inputenc}
\usepackage{fancyhdr}
\usepackage{lastpage}
\usepackage{listings}
\usepackage{amssymb}
\usepackage[usenames,dvipsnames]{color}
\usepackage{graphicx}
\usepackage{wrapfig}
\usepackage{amsmath}
\usepackage{makeidx}
\usepackage{color}
\usepackage{currfile}
\usepackage{xcolor}

% Margenes
\setlength{\columnsep}{0.4in}
\setlength{\columnseprule}{0.1pt}
\addtolength{\textheight}{2.9in}
\addtolength{\topmargin}{-1.2in}
\addtolength{\textwidth}{1.8in}
\addtolength{\oddsidemargin}{-0.9in}
\setlength{\headsep}{0.08in}
\setlength{\parskip}{0in}
\setlength{\headheight}{15pt}
\setlength{\parindent}{0mm}

% Encabezado de pagina
\pagestyle{fancy}
\fancyhead[LO]{\textbf{\title}}
\fancyhead[C]{\leftmark}
\fancyhead[RO]{P\'agina \thepage\ de \pageref{LastPage}}

% Mostrar codigo
\lstloadlanguages{C++}

\lstnewenvironment{code}
	{\csname lst@SetFirstLabel\endcsname}
	{\csname lst@SaveFirstLabel\endcsname}

\lstset{
    language=C++,
    basicstyle=\small\ttfamily, % Set the font style for the code
    keywordstyle=\color{blue}, % Set the color for keywords
    stringstyle=\color{green!60!black}, % Set the color for strings
    commentstyle=\color{red}, % Set the color for comments
    breaklines=true, % Allow breaking lines within the code
    postbreak=\mbox{\textcolor{red}{$\hookrightarrow$}\space}, % Symbol at the end of a broken line
    showstringspaces=false % Don't show spaces in strings as underscores
}

\begin{document}
\def\title{Notebook}
\tableofcontents\newpage


\section{Setup}
    \subsection{Template}
        \cppfile[1-21]{setup/template.cpp}
    \subsection{Makefile}
        \begin{code}
CPPFLAGS = -std=c++17 -O0 -Wall -g
CC = g++
\end{code}
    \subsection{Compilar}
        Compilar \$1 y mostrar primeras \$2 lineas de error
        \begin{code}
rm -f $1
clear
make $1 2>&1 | head -$2
\end{code}
    \subsection{Correr}
        Correr \$1 con el input \$2
        \begin{code}
clear
make -s $1 && ./$1 < $2
\end{code}

\section{STL}
    \subsection{Búsqueda binaria en vector ordenado}
        \cppfile[1-99]{stl/busqueda_binaria.cpp}
    \subsection{Priority queue custom compare}
        \cppfile[1-99]{stl/priority_queue_custom.cpp}
    \subsection{Intervalos consecutivos}
        \cppfile[1-99]{stl/intervalos_consecutivos.cpp}
    \subsection{Indexed set y multiset}
        \cppfile[1-99]{stl/indexed_set.cpp}
        
\section{Range queries}
\subsection{Prefix + diff arrays}
    \subsubsection{1D}
        Usar array indexado desde $1$ con \texttt{A[0] = 0}. \\
        Usar intervalos cerrado-cerrado (indexados desde $1$).
        \cppfile[1-99]{range-queries/prefix-diff/1D.cpp}
    \subsubsection{2D}
        % \cppfile[1-99]{range-queries/prefix-diff/2D.cpp}
\subsection{Fenwick tree}
    \subsubsection{Range query point update}
        % \cppfile[1-99]{estructuras_rango_grupo/.cpp}
    \subsubsection{Range update point query}
        % \cppfile[1-99]{estructuras_rango_grupo/.cpp}
    \subsubsection{Range update range query}
        % \cppfile[1-99]{estructuras_rango_grupo/.cpp}
\subsection{Operaciones sin inverso}
    \subsubsection{Sparse table}
        Operacion asociativa \textbf{idempotente}
        \cppfile[1-99]{estructuras_rango_monoide/sparse_table.cpp}
    \subsubsection{Segment tree: range query point set}
        Recordatorio: modificar elemento neutro
        \cppfile[1-99]{estructuras_rango_monoide/segment_tree.cpp}
       
\section{Grafos}
\subsection{Preprocesamiento}
    \subsubsection{Clasificación de aristas}
%       \cppfile[1]{preprocesamiento_grafos/.cpp}
    \subsubsection{Puentes y puntos de articulación}
%       \cppfile[1]{preprocesamiento_grafos/.cpp}
    \subsubsection{DAG condensado}
%       \cppfile[1]{preprocesamiento_grafos/.cpp}
    \subsubsection{Kruskal}
%       \cppfile[1]{preprocesamiento_grafos/.cpp}
\subsection{Menor camino}
    \subsubsection{BFS}
        % \cppfile[1]{menor_camino_grafo/.cpp}
    \subsubsection{Dijkstra}
        % \cppfile[1]{menor_camino_grafo/.cpp}
    \subsubsection{Floyd-Warshall}
        % \cppfile[1]{menor_camino_grafo/.cpp}
\subsection{Flujo y corte}
    \subsubsection{Dinics}
        % \cppfile[1]{/.cpp}
    \subsubsection{Maximum matching}
        % \cppfile[1]{/.cpp}
\subsection{Árboles}
    \subsubsection{Aplanamiento}
        % \cppfile[1]{/.cpp}
    \subsubsection{Ancestro común menor}
        % \cppfile[1]{/.cpp}
        
\section{Strings}
    \subsection{Trie: policy based}
        % \cppfile[1]{/.cpp}
    \subsection{Trie genérico}
        % \cppfile[1]{/.cpp}
    \subsection{Rabin-Karp}
        % \cppfile[1]{/.cpp}
    \subsection{Suffix Automata}
        % \cppfile[1]{/.cpp}

\section{Matemática}
\subsection{Aritmética}
    \subsubsection{Techo de la división, piso de la raiz cuadrada, piso del log2}
        \cppfile[1]{aritmetica/aritmetica.cpp}
    \subsubsection{Aritmética en Zp}
        \cppfile[1]{aritmetica/aritmetica_zp.cpp}
    \subsubsection{Números combinatorios}
        % \cppfile[9-9]{aritmetica/.cpp}
\subsection{Teoría de números}
\subsubsection{Test de primalidad}
    \cppfile[1-99]{matematica/primetest.cpp}
\subsection{Geometría}
    \subsubsection{Template base}
        \cppfile[1]{geometria/template_base.cpp}
    \subsubsection{Punto/vector/recta}
        % \cppfile[1]{/.cpp}
    \subsubsection{Producto escalar y vectorial}
        % \cppfile[1]{/.cpp}
    \subsubsection{Área triángulo}
        % \cppfile[1]{/.cpp}
    \subsubsection{Fórmula de Herón}
    % \cppfile[1]{/.cpp}
    
\section{Programación Dinámica}
\subsection{Ejemplos de DP}
    \subsubsection{DP en prefijo: }
        % \cppfile[1]{/.cpp}
    \subsubsection{DP en rango: }
        % \cppfile[1]{/.cpp}
    \subsubsection{DP en bitmask: traveling salesman}
        % \cppfile[1]{/.cpp}
    \subsubsection{DP en árbol con toposort}
        % \cppfile[1]{/.cpp}
    \subsubsection{DP en DAG: }
        % \cppfile[1]{/.cpp}
    \subsubsection{DP en número: knapsack}
    % \cppfile[1]{/.cpp}
    \subsubsection{Re-rooting DP}
        % \cppfile[1]{/.cpp}
    \subsubsection{DP con Fenwick: número de subsecuencias de largo k}
        % \cppfile[1]{/.cpp}
    \subsubsection{Reconstruir solución}
        % \cppfile[1]{/.cpp}
\subsection{Optimizaciones}
    \subsubsection{Recuperar un parámetro a partir de los otros}
        % \cppfile[1]{/.cpp}
    \subsubsection{Reducir complejidad de transición con una flag}
        % \cppfile[1]{/.cpp}
    \subsubsection{Optimización knapsack en árbol}
        % \cppfile[1]{/.cpp}
    \subsubsection{Optimización de Knuth}
        % \cppfile[1]{/.cpp}
    \subsubsection{Optimización D\&C}
        % \cppfile[1]{/.cpp}

\section{Algoritmos}
    \subsection{Búsqueda binaria}
        Si existe, idx de primer \texttt{true}
        Si no, \texttt{d}
        \cppfile[1-99]{algoritmos/busq_binaria.cpp}
    \subsection{Búsqueda binaria paralela}
    %   \cppfile[1-99]{algoritmos/.cpp}

\section{Sin categorizar}
    \subsection{Union find}
    \subsection{Algoritmo de Mo}
    \subsection{Min dequeue}
    \subsection{Menor subarray que suma k}
    \subsection{Subarray con mayor suma}
    \subsection{Mayor subcadena común}

\section{Brainstorming}
Graficar como puntos/grafos \\
Pensarlo al revez \\
¿Que propiedades debe cumplir una solución? \\
Si existe una solución, ¿existe otra más simple? \\
¿Hay elecciones independientes? \\
¿El proceso es parecido a un algoritmo conocido? \\

\end{document}